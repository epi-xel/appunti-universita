\documentclass[a4paper]{article}
\usepackage[T1]{fontenc} 
\usepackage[utf8]{inputenc}
\usepackage[italian]{babel}
\usepackage{amsmath}
\usepackage{amsthm}
\usepackage{xcolor}
\usepackage{framed}
\usepackage{lipsum}
\usepackage{geometry}
\usepackage{graphicx}
\date{Appello di Giugno 2020}
\author{Elisa Pioldi\\mat. 856591}
\geometry{a4paper, top=3cm, bottom=3cm, left=2.5cm, right=2.5cm, heightrounded, bindingoffset=5mm}
\definecolor{shadecolor}{gray}{0.80}
\title{\textbf{ESERCIZIO PREMIANTE\\Architettura degli Elaboratori}}
%
\begin{document}
%
\maketitle
%
\section{Esecuzione in QtSpim}
Il codice proposto è il seguente:
\begin{verbatim}
main:
beq $0,$0, realMain
realMain:
jr $31
\end{verbatim}
e viene eseguito in QtSpim senza particolari problemi. Le istruzioni vengono assemblate in questo modo:
\begin{verbatim}
[00400024] 10000001  beq $0, $0, 4 [realMain-0x00400024]
[00400028] 03e00008  jr $31                   ; 4: jr $31 
\end{verbatim}
Si nota pertanto che nel momento in cui la branch viene effettuata, l'offset risulta pari a \texttt{4}, dovendo saltare all'istruzione successiva etichettata da {\itshape RealMain}.

L'esecuzione delle istruzioni preliminari nell'\texttt{User Text Segment} di SPIM, non facenti parte nel nostro codice effettivo, assicura che il programma nel momento dell'istruzione \texttt{jr} salti al registro \texttt{\$31}, corrispondente al {\itshape register address}, e con l'esecuzione dell'istruzione riferita dall'indirizzo contenuto in \texttt{\$ra}, il {\itshape register address}, si conclude per un'opportuna syscall finale.
\begin{verbatim}
[00400018] 00000000  nop                      ; 189: nop 
[0040001c] 3402000a  ori $2, $0, 10           ; 191: li $v0 10 
[00400020] 0000000c  syscall                  ; 192: syscall # syscall 10 (exit) 
\end{verbatim}
Durante l'esecuzione il ramo branch viene effettuato e il programma si conclude in QtSpim.
%
\section{Esecuzione in Logisim}
Dopo aver salvato le istruzioni assemblate in un file di testo con estensione \texttt{.img}:
\begin{verbatim}
v2.0 raw
10000001
03e00008
\end{verbatim}
Le carichiamo nella memoria del circuito simulazione del Datapath Muticiclo (figura \ref{fig1}).

Procedendo quindi con l'esecuzione delle istruzioni notiamo subito che il comportamento si rivela inatteso.
Per prima cosa mancano le istruzione dell'\texttt{User Text Segment}, quindi il programma continua a eseguire le istruzioni anche nulle indefinitamente. Inoltre il {\itshape register address} non contiene il valore di default di SPIM che fa eseguire la conclusione del programma, ma il contenuto è nullo.

Ci si aspetterebbe quindi che dopo l'istruzione jr venga eseguta l'istruzione contenuta all'indirizzo segnato dal {\itshape register address}, l'istruzione di indirizzo nullo, ovvero la {\itshape branch}. Ciò che si verificherebbe sarebbe un ciclo infinito.

Avanzando con i cicli di clock, osserviamo dunque che il {\itshape program counter} viene incrementato, ma controllando l'{\itshape instruction register} l'istruzione jr codificata come \texttt{03e00008} non viene mai eseguita, passando dall'istruzione di {\itshape branch} a quelle nulle successive a \texttt{jr}, e così via (figura \ref{fig2}).
In un primo momento sembra pertanto che durante la {\itshape branch} occorra un problema.

Provando quindi a caricare in Logisim alcune istruzioni assemblate simili, ma tali che con la {\itshape branch} non venga effettuato l'offset
\begin{verbatim}
main:
li $t0, 2
beq $t0, $0, realMain
realMain:
jr $31
\end{verbatim}
si vede che il programma non presenta problemi, e l'istruzione \texttt{jr} viene effettuata.

Se ne deduce quindi che il problema risiede nella consecutività dell'istruzione \texttt{jr} quando viene valutato il ramo della {\itshape branch}.
Osservando attentamente il {\itshape program counter} si nota che, come dovrebbe essere, ad ogni {\itshape fetch} viene incrementato di \texttt{4}. La discordanza va ricercata nell'istruzione assemblata da SPIM di {\itshape branch} presenta un offset di \texttt{4} che va a sommarsi al già incrementato {\itshape program counter}, causando quindi un salto di un'istruzione, in questi caso la \texttt{jr}. Se il ramo della {\itshape branch} non venisse valutato, allora il {\itshape program counter} verrebbe incrementato solo di \texttt{4} e inoltre l'istruzione \texttt{jr} verrebbe eseguita.
\begin{figure}
\centering
\caption{Istruzioni caricate in Logisim}
\includegraphics[width=.8\textwidth, height=.6\textheight, keepaspectratio]{ram.jpg}
\label{fig1}
\end{figure}
\begin{figure}
\centering
\caption{Instruction Register}
\includegraphics[width=.8\textwidth, height=.6\textheight, keepaspectratio]{instre.jpg}
\label{fig2}
\end{figure}
%
\section{Correzione delle istruzioni e confronto con SPIM}
Per correggere le istruzioni assemblate, dal momento che i problemi sono presenti nel ramo della {\itshape branch}, è necessario quindi operare sull'offset relativo a questo: tenendo presente che il {\itshape program counter} verrà incrementato di \texttt{4}, per passare all'istruzione successiva l'offset dovrà essere di \texttt{0}.

I byte modificati risulteranno quindi come i seguenti:
\begin{verbatim}
10000000
03e00008
\end{verbatim}

Inserendoli come immagine nella memoria del circuito, si vedrà che non ci saranno problemi, l'istruzione \texttt{jr} verrà eseguita e avrà luogo un ciclo infinito.

Resta un'ultima osservazione da fare al riguardo: il simulatore QtSpim nell'assemblare le istruzioni sembra non tenere conto dell'incremento del {\itshape program counter}, in quanto l'istruzione viene assemblata secondo un offset di \texttt{4} e non di \texttt{0}.

\end{document}
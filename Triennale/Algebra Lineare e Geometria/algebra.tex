\documentclass[a4paper]{report}
\usepackage[T1]{fontenc} 
\usepackage[utf8]{inputenc}
\usepackage[italian]{babel}
\usepackage{amsmath}
\usepackage{amsthm}
\usepackage{xcolor}
\usepackage{framed}
\usepackage{lipsum}
\usepackage{geometry}
\geometry{a4paper, top=3cm, bottom=3cm, left=2.5cm, right=2.5cm, heightrounded, bindingoffset=5mm}
%\usepackage{imakeidx}
\definecolor{shadecolor}{gray}{0.80}
\newtheorem{teoremaDet}{Teorema}[section]
\theoremstyle{remark}
\newtheorem{nota1}{Nota}
\newtheorem{oss}{Osservazioni}
\theoremstyle{definition}
\newtheorem{detProp}{Proprietà del determinante}
\newtheorem{detDef}{Definizione}
\newtheorem{es}{Esempio}
\newcommand{\uvec}[2][3]{\underline{#2\mkern-#1mu}\mkern#1mu}
%\makeindex
\title{Algebra Lineare e Geometria Analitica}
%
\begin{document}
%
\maketitle
%
\tableofcontents
%
\pagenumbering{arabic}
\pagestyle{headings}
%
\chapter{Spazi Vettoriali}
\chapter{Associare Vettori a Punti}
\chapter{Somma di Vettori}
\chapter{Sottospazi Vettoriali}
\chapter{Dipendenza Lineare e Basi}
\chapter{Matrici}
%
\chapter{Il Determinante}
%
\section{Definizione e prime proprietà}
\label{detProp}

\begin{shaded}
\begin{detDef}
	Come oggetto matematico il determinante è una funzione  
	\[
	{det_n: \quad \{Matrice\quad n\times n \} \longmapsto \Re }
	\]
\end{detDef}
\end{shaded}
%
Ci sono bijezioni tra:

$\{Matrice \quad n\times n \} \iff \{(\vec{c_1}, \dots{} , \vec{c_n}),\quad \vec{c_i}$ = vettori colonna di $\Re^{n}\}$

$\{Matrice \quad n\times n \} \iff \{(\vec{r_1}, \dots{} , \vec{r_n}),\quad \vec{r_i}$ = vettori riga di $\Re^{n}\}$

\bigskip
L'importanza del $det_n$ è che trova l'invertibilità della matrice:
$det(M) \ne 0\quad sse$ $M$ è invertibile. Inoltre, c'è un algoritmo che calcola $det_n(M)$. Si noti che $det_n$ fornisce un modo di mostrare che una matrice è invertibile senza trovare la matrice inversa.
\begin{teoremaDet}
	Esiste una sola funzione
	\[
	f: (\Re^{n})^{n} = \{vettori colonna in \Re ^{n}\} \longmapsto \Re
	\]
	che soddisfa le proprietà $1,2,3,4$. Questa funzione è il determinante $det_n$.
\end{teoremaDet}
%
\begin{detProp}
($\vec{v}_1,\dots,\vec{v}_n$ denotano vettori colonna di $\Re^{n}$ e $\lambda \in \Re$.)

\begin{enumerate}
	\item $det_n(\uvec{v}_1,\dots,\uvec{v}_i+\uvec{w},\,\uvec{v}_{i+1}\dots{v}_n)\,=\, 			det_n(\uvec{v}_1,\dots,\uvec{v}_i+\dots,\uvec{v}_n)\,+ \, det_n(\uvec{v}_1,\dots,\uvec{w},\,\uvec{v}_n)\\
	\forall i=1,\dots n$
	\item $det_n(\uvec{v}_1,\dots,\lambda\,\uvec{v}_i,\,\uvec{v}_{i+1},\dots\uvec{v}_n)\,=\, \lambda\,det_n(\uvec{v}_1,\dots,\uvec{v}_i+\dots,\uvec{v}_n)\\
	\forall i=1,\dots n$
	\item $det_n(\uvec{v}_1,\dots,\uvec{w},\,\uvec{w},\,\uvec{v}_{i+2},\dots\uvec{v}_n)\,=\,0\\
	\forall i=1,\dots n$
	\item $det_n(Id_n)\,=\,1$,
	dove\quad $Id\,=$
	$
	\begin{pmatrix}
	1 & 0 & \dots & 0\\
	0 & 1 & \dots & 0\\
	\vdots & 0 & \dots & 0\\
	\vdots & \vdots & \ddots & 0\\
	0 & 0 & \dots & 1
	\end{pmatrix}
	$
\end{enumerate}
\end{detProp}
%
\begin{nota1}
%\begin{itemize}
	Le proprietà {\bfseries 1} e {\bfseries 2} sono la {\itshape multilinearità di $n$}\\
	La proprietà {\bfseries 3} è chiamata {\itshape proprietà alternante}; tale proprietà per una funzione multilineare in $n$ è equivalente a:\\
	$det_n(\vec{v}_1,\dots,\vec{v}_i+,\,\vec{v}_{i+1}\dots{v}_n)\,=\,-\,det_n(\vec{v}_1,\dots,\vec{v}_{i+1}+,\,\vec{v}_i\dots{v}_n)$
%\end{itemize}
\end{nota1}

\begin{proof}
\label{dimDet2}
	(del teorema precedente del caso speciale di $n\,=\,2$). Supponiamo che $f$ soddisfi {\bfseries 1}, {\bfseries 2}, {\bfseries 3}, {\bfseries 4}. Calcoliamo $f\,\bigl(
	\begin{smallmatrix}
		a & b \\
		c & d
	\end{smallmatrix}
	\bigr)$
	
	\medskip
	% inizia la dimostrazione
	$
%	\begin{split}
	f
	\begin{pmatrix}
		a & b\\
		c & d
	\end{pmatrix}
	= \,f
	\begin{pmatrix}
		\begin{pmatrix}
			a\\
			c
		\end{pmatrix}
		,
		\begin{pmatrix}
			b\\
			d
		\end{pmatrix}
	\end{pmatrix}
	= \,f
	\begin{pmatrix}
		\begin{pmatrix}
			a\\
			0
		\end{pmatrix}
		+
		\begin{pmatrix}
			0\\
			c
		\end{pmatrix}
		,
		\begin{pmatrix}
			d\\
			c
		\end{pmatrix}
	\end{pmatrix}
	\stackrel{(1)}{=}\,f
	\begin{pmatrix}
		\begin{pmatrix}
			a\\
			0
		\end{pmatrix}
		,
		\begin{pmatrix}
			b\\
			d
		\end{pmatrix}
	\end{pmatrix}
	+\,f
	\begin{pmatrix}
		\begin{pmatrix}
			0\\
			c
		\end{pmatrix}
		,
		\begin{pmatrix}
			b\\
			d
		\end{pmatrix}
	\end{pmatrix}\\
	=\,f
	\begin{pmatrix}
		a
		\begin{pmatrix}
			1\\
			0
		\end{pmatrix}
		,
		\begin{pmatrix}
			b\\
			d
		\end{pmatrix}
	\end{pmatrix}
	+\,f
	\begin{pmatrix}
		c
		\begin{pmatrix}
			0\\
			1
		\end{pmatrix}
		,
		\begin{pmatrix}
			b\\
			d
		\end{pmatrix}
	\end{pmatrix}
	\stackrel{(2)}{=}\,
	a\,f
	\begin{pmatrix}
		\begin{pmatrix}
			1\\
			0
		\end{pmatrix}
		,
		\begin{pmatrix}
			b\\
			d
		\end{pmatrix}
	\end{pmatrix}
	+ c\,f
	\begin{pmatrix}
		\begin{pmatrix}
			0\\
			1
		\end{pmatrix}
		,
		\begin{pmatrix}
			b\\
			d
		\end{pmatrix}
	\end{pmatrix}\\
	\stackrel{(1)}{=}\,a
	\begin{bmatrix}
		f
		\begin{pmatrix}
			\begin{pmatrix}
				1\\
				0
			\end{pmatrix}
			,
			\begin{pmatrix}
				b\\
				0
			\end{pmatrix}
		\end{pmatrix}
		+\,f
		\begin{pmatrix}
			\begin{pmatrix}
				1\\
				0
			\end{pmatrix}
			,
			\begin{pmatrix}
				0\\
				d
			\end{pmatrix}
		\end{pmatrix}
	\end{bmatrix}
	+\,c
	\begin{bmatrix}
		f
		\begin{pmatrix}
			\begin{pmatrix}
				0\\
				1
			\end{pmatrix}
			,
			\begin{pmatrix}
				b\\
				0
			\end{pmatrix}
			\end{pmatrix}
		+\,f
		\begin{pmatrix}
			\begin{pmatrix}
				0\\
				1
			\end{pmatrix}
			,
			\begin{pmatrix}
				0\\
				d
			\end{pmatrix}
		\end{pmatrix}
	\end{bmatrix}\\
	\stackrel{(2)}{=}\,a
	\begin{bmatrix}
		b\,\underbrace {f
		\begin{pmatrix}
			\begin{pmatrix}
			1\\
			0
			\end{pmatrix}
			,
			\begin{pmatrix}
			1\\
			0
			\end{pmatrix}
		\end{pmatrix}}_
		{\stackrel{(2)}{=}\,0}
		+\,d\,f
		\begin{pmatrix}
			\begin{pmatrix}
				1\\
				0
			\end{pmatrix}
			,
			\begin{pmatrix}
				0\\
				1
			\end{pmatrix}
		\end{pmatrix}
		\end{bmatrix}+\,c
		\begin{bmatrix}
		b\,f
		\begin{pmatrix}
			\begin{pmatrix}
				0\\
				1
			\end{pmatrix}
			,
			\begin{pmatrix}
				1\\
				0
			\end{pmatrix}
		\end{pmatrix}
		+\,d\,f\underbrace{
		\begin{pmatrix}
			\begin{pmatrix}
			0\\
			1
			\end{pmatrix}
			,
			\begin{pmatrix}
			0\\
			1
			\end{pmatrix}
		\end{pmatrix}}_
		{\stackrel{(2)}{=}\,0}
	\end{bmatrix}\\
	= \,a\,d\,f
	\begin{pmatrix}
		\begin{pmatrix}
			1\\
			0
		\end{pmatrix}
		,
		\begin{pmatrix}
			0\\
			1
		\end{pmatrix}
	\end{pmatrix}
	+ \,c\,b\underbrace{f
	\begin{pmatrix}
		\begin{pmatrix}
		0\\
		1
		\end{pmatrix}
		,
		\begin{pmatrix}
		1\\
		0
		\end{pmatrix}
	\end{pmatrix}}_
	{\stackrel{(3)}{=}-f
	\begin{pmatrix}
		\begin{pmatrix}
		1\\
		0
		\end{pmatrix}
		,
		\begin{pmatrix}
		0\\
		1
		\end{pmatrix}
	\end{pmatrix}}
	\stackrel{(4)}{=}\,ad\,-\,cb
%	\end{split}
	$
\end{proof}
\bigskip
\section{Teorema di Laplace}
\label{laplace}

\begin{shaded}
	\begin{teoremaDet}[di Laplace]
	\label{Laplace}
		Sia A\,=\,$(a_{ij})$ una matrice $n \times n$, e sia $(a_{k1},\,a_{k2},\dots,a_{kn})$ la sua $k$-esima riga. Allora abbiamo la seguente uguaglianza:
		\[
		det(A)\,=\,\sum_{j=1}^n(-1)^{k+j}\,\,a_{kj}\,\,det(a_{kj})
		\]
		dove $A_{kj}\,=\,
		\begin{pmatrix}
		a_{11} & a_{11} & \dots & a_{1n}\\
		a_{21} & \dots & \dots & a_{2n}\\
		\vdots &  &  & \vdots\\
		a_{k1} & \dots & a_{kj} & a_{kn}\\
		\vdots &  &  & \vdots\\
		a_{n1} & \dots & \dots & a_{1n}
		\end{pmatrix}
		$
	\end{teoremaDet}
\end{shaded}
%
\begin{oss}
		$^{(1)}$La formula dipende dalla riga $k$, ma il $det(a)$ no.
		
		$^{(2)}$C'è uno sviluppo simile del determinante attraverso una $h$-esima colonna invece della $k$-esima riga:
		\[
		det(A)\,=\,\sum_{i=1}^n(-1)^{i+h}\,\,a_{ih}\,\,det(a_{ih})
		\]
		
		$^{(3)}$Queste sono definizioni ricorsive, ciò significa che il determinante di una matrice $n\times n$ è espresso in funzione del determinante della matrice $(n-1)\times(n-1)$
\end{oss}
\begin{es}
	Dato $det(a)\,=\,a$, si scriva la formula per il determinante di una generica matrice $2\times 2$ e si controlli che coincida con quello derivato nella dimostrazione \ref{dimDet2}.
	
	Sia $A\,=\,\bigl(
	\begin{smallmatrix}
	a & b\\
	c & d
	\end{smallmatrix}
	\bigr)$, con la formula di Laplace abbiamo, sviluppando secondo la prima riga,
	
	$det
	\begin{pmatrix}
	a & b\\
	c & d
	\end{pmatrix}\,=\,(-1)^{1+1}\,a\,det(d)\,+\,(-1)^{1+2}\,b\,det(c)\,=\,ad-bc$
\end{es}
\begin{es}
	Si calcoli il determinante si $A\,=\,
	\begin{pmatrix}
		2 & 0 & 1\\
		1 & 1 & 0\\
		-1 & 3 & 2
	\end{pmatrix}$
	
	Sviluppiamo di nuovo il determinante secondo una riga. Si noti che è sempre conveniente scegliere una riga con il più alto numero di zeri. Perciò scegliamo o la prima o la seconda riga. Prendiamo la seconda:
	
	$det\,\begin{pmatrix}
		2 & 0 & 1\\
		1 & 1 & 0\\
		-1 & 3 & 2
	\end{pmatrix}
	\,=\,(-1)^{2+1}\,1\,det
	\begin{pmatrix}
		0 & 1\\
		3 & 2\\
	\end{pmatrix}
	\,+\,(-1)^{2+2}\,1\,det
	\begin{pmatrix}
		2 & 1\\
		-1 & 2\\
	\end{pmatrix}\\
	=\,-(-3)+4+1=8$
\end{es}
%
\section{Altre proprietà e teorema di Binet}
\begin{detProp}
	La funzione determinante, oltre alle proprietà definite (vedi \ref{detProp}):
	\begin{itemize}
		\item Multilinearità in $n$
		\item Alternanza
		\item $det(Id)=1$
	\end{itemize}
	ha le seguenti:
	\begin{enumerate}
		\item $det(A)\,=\,det(^{t}A)$, dove $^{t}A$ è la matrice trasposta di A.
		\item Se A è una matrice {\itshape triangolare}, cioè
	
			$A\,=\,
			\begin{pmatrix}
				a_{11} & \dots & 0\\
				\vdots & \ddots & \vdots\\
				* & \dots & a_{nn}
			\end{pmatrix}\quad$
			o
			$\quad A\,=\,
			\begin{pmatrix}
			a_{11} & \dots & *\\
			\vdots & \ddots & \vdots\\
			0 & \dots & a_{nn}
			\end{pmatrix}$,
			
			allora $det(A)\,=\,a_{11}\,a_{22}\dots a_{nn}$
			\begin{proof}
				Per induzione sull'ordine $n$ della matrice:
				
				Base: $n=2$, $det\,\bigr(
				\begin{smallmatrix}
				a_{11} & 0\\
				a_{21} & a_{22}
				\end{smallmatrix}
				\bigl)\,=\,a_{11}\,a_{22}$ (vedi \ref{detProp} e \ref{laplace})
				
				Passo induttivo: supponiamo che l'asserzione sia valida per ogni matrice fino all'ordine $n-1$.
				
				Per la formula di Laplace (\ref{laplace}):
				
				$det\,
				\begin{pmatrix}
				a_{11} & \dots & \dots & 0\\
				\vdots & a_{22} & & \vdots\\
				\vdots & & \ddots & \vdots\\
				* & \dots & \dots & a_{nn}
				\end{pmatrix}
				\,=\,a_{11}\,det\,
				\begin{pmatrix}
				a_{22} & \dots & 0\\
				\vdots & \ddots & \vdots\\
				* & \dots & a_{nn}
				\end{pmatrix}$
				
				${\stackrel{per\,ipotesi\,induttiva}{=}}\,a_{11}\,(a_{22}\dots a_{nn})$
				
			\end{proof}
		\item In generale $det\,(A+B)\,\ne\,det(A)\,+\,det(B)$. Comunque, 
			\begin{teoremaDet}[di Binet]
				$det(A\,B)\,=\,det(A)\,det(B)$
			\end{teoremaDet}
		\item È conseguenza immediata del teorema di Binet:
			\begin{teoremaDet}[corollario]
				Se A è invertibile, allora $det(A^{-1})=\dfrac{1}{det(A)}$
			\end{teoremaDet}
			\begin{proof}
				Applichiamo il teorema di Binet a $Id_{n} = A^{-1}A:$
				\[
				1=det(Id_{n})=det(A^{-1}A){\stackrel{Binet}{=}}det(A^{-1}detA)
				\]
			\end{proof}
		\item La relazione tra operazioni elementari e il determinante richiama le tre operazioni elementari sulle righe e 	sulla matrice:
			\begin{enumerate}
				\item [{\itshape a.}] scambiare le due righe
				\item [{\itshape b.}] moltiplicare una riga per $\lambda\ne0$
				\item [{\itshape c.}] sostituire $r_{i}$ con $r_{i}+\alpha r_{j}$.
			\end{enumerate}
			Cosa succede al determinante dopo che applichiamo un'operazione elementare?
			\begin{itemize}
				\item {\itshape Operazione a.} Il determinante è moltiplicato per $-1$. Per vedere questo, si noti che scambiare $r_{i}$ e $r_{j}$ ($j>i$) è lo stesso che fare $j-i-1$ scambi {\em consecutivi} su $r_{i}$ e $j-i-1$ scambi {\em consecutivi} su $r_{j}$. Quindi scambiamo {\em consecutivamente} un numero dispari di volte. Ora applichiamo l' ``equivalente'' proprietà alternante del determinante (vedi \ref{detProp}).
				\item {\itshape Operazione b.} Il determinante resta invariato:
				
				$det(\dots,r_{i}+\alpha r_{j},\dots)\,=\,det(\dots,r_i,\dots)+\alpha\,det(\dots,r_j,\dots,r_j,\dots)$
				
				Si noti che:
				
				$det(\dots,r_{i}+ r_{j},\dots)\,=\,(-1)^{j-i-1}\,\underbrace{det(\dots,r_j,r_j,\dots)}_{=0}$ per l'operazione {\itshape a}. 
				
				Ma $det(\dots,r_j,r_j,\dots)=0$ per l'alternanza di {\itshape c.} (vedi \ref{detProp}).
				
				Così abbiamo provato che $det(\dots,r_{i}+\alpha r_{j},\dots)\,=\,det(A)$
			\end{itemize}
	\end{enumerate}
\end{detProp}
%
\section{Determinante e sistemi di equazioni lineari}
%
Sia $A\uvec{x}=\uvec{b}$ un sistema di equazioni lineari con $A$ una matrice quadrata (vedi 6.4). Allora:
\begin{shaded}
\begin{teoremaDet}
	\begin{enumerate}
		\item Il sistema ammette una soluzione sse $ det(A)\ne0 $
		\item In quel caso la soluzione $ (c_1,\dots,c_n) $ è unica e tale che 
		\[
		c_i\,=\,\dfrac{det(A_1\dots|\uvec{b}|\dots A_n)}{det(A)}
		\]
		dove $ A_j $ è il $ j $-esimo vettore colonna di $ A $.
	\end{enumerate}
\end{teoremaDet}
\end{shaded}
\begin{es}
	Controlliamo se $
	\begin{cases}
		2x+z-2=0\\
		-y-3z+2=0\\
		-x+y-z=0
	\end{cases}
	$ ha un'unica soluzione e in quel caso troviamola.
	
	La matrice $ A $ è $
	\begin{pmatrix}
		2 & 0 & 1\\
		0 & -1 & 3\\
		-1 & 1 & -1
	\end{pmatrix}$
	e $ \uvec b\,=\,$
	$\begin{pmatrix}
		2\\-2\\0
	\end{pmatrix}$
	
	$ det(A)\,=\,2\,det $
	$\begin{pmatrix}
		-1 & -3\\
		1 & -1
	\end{pmatrix}$
	$\,+\, det$
	$\begin{pmatrix}
		0 & -1\\
		-1 & 1
	\end{pmatrix}$
	$ \,=\,2(1+3)+(-1)=7\, \Rightarrow\exists !$  soluzione
	
	Per il teorema precedente, possiamo esprimere la soluzione come $\uvec{c}=(c_1,c_2,c_3)$ con:
	
	$ c1\,=\,\dfrac{det
	\begin{pmatrix}
		2 & 0 & 1\\
		-2 & -1 & -3\\
		0 & 1 & -1
	\end{pmatrix}}{7} 
	\,=\,\dfrac{1}{7}\Bigl( 2\,det
	\begin{pmatrix}
		-1 & -3\\
		1 & -1
	\end{pmatrix}
	\,+\,det
	\begin{pmatrix}
		-2 & -1\\
		0 & 1
	\end{pmatrix}\Bigr)
	= \dfrac{1}{7}(8-2)=\dfrac{6}{7}$
	
	$ c1\,=\,\dfrac{det
	\begin{pmatrix}
		2 & 2 & 1\\
		0 & -2 & -3\\
		-1 & 1 & -1
	\end{pmatrix}}{7} 
	\,=\,\dfrac{1}{7}\Bigl( 2\,det
	\begin{pmatrix}
		-2 & -3\\
		0 & -1
	\end{pmatrix}
	\,-\,det
	\begin{pmatrix}
		2 & 1\\
		-2 & -3
	\end{pmatrix}\Bigr)
	= \dfrac{1}{7}(4+4)=\dfrac{8}{7}$
	
	$ c1\,=\,\dfrac{det
	\begin{pmatrix}
		2 & 0 & 2\\
		0 & -1 & -2\\
		-1 & 1 & 0
	\end{pmatrix}}{7} 
	\,=\,\dfrac{1}{7}\Bigl( 2\,det
	\begin{pmatrix}
		-1 & 2\\
		1 & 0
	\end{pmatrix}
	\,+\,det
	\begin{pmatrix}
		0 & -1\\
		-1 & 1
	\end{pmatrix}\Bigr)
	= \dfrac{1}{7}(4-2)=\dfrac{2}{7}$
\end{es}
%
\section{Determinante e rango di una matrice}
Il determinante è definito solo per le matrici quadrate. Comunque, è possibile usare il concetto di determinante per ottenere informazioni riguardo qualunque matruce.
\begin{detDef}
	Sia $ A $ una matrice. Una sottomatrice di $ A $ è una matrice ottenuta togliendo alcune righe e alcune colonne da $ A $.
\end{detDef}
\begin{detDef}
	Un \textbf{minore} di ordine $ k $ di una matrice $ A $ è il determinante di una sottomatrice quadrata di $ A $ di ordine $ k $.
\end{detDef}
\begin{es}
	Sia $ A $ la matrice $ 
	\begin{pmatrix}
		3 & 0 & -1 & 2\\
		1 & 2 & 2 & 1\\
		0 & -1 & -1 & 2
	\end{pmatrix} $.
	Una sottomatrice quadrata di ordine 2 è $ B= 
	\begin{pmatrix}
		3&-1\\
		0&-1
	\end{pmatrix}$.
	Il suo minore associato è $ det(B)=-3 $.
\end{es}
\begin{teoremaDet}
	Sia $A$ una qualunque matrice. Allora il rango di $A$ è uguale al più grande ordine di minori non nulli di $A$.
\end{teoremaDet}
\begin{es}
	Consideriamo la matrice $ A $ scritta sopra: è una matrice $ 3 \times 4 $, perciò $ rank(A) \geq 2 $ per il teorema.
	
	Ci sono due possibilità:
	\begin{enumerate}
		\item $ \exists $ minore non nullo di ordine 3 di $ A $
		\item Tutte le matrici di ordine 3 di $ A $ hanno $ det=0 $
	\end{enumerate}

	In $ A $ ci sono quattro sottomatrici quadrate di ordine 3. Comunque non è necessario controllare il determinante di tutte queste: per il teorema (o anche solo osservandole), abbiamo che le colonne $ 
	\begin{pmatrix}
		3\\
		1\\
		0
	\end{pmatrix} $ e $
	\begin{pmatrix}
		-1\\
		2\\
		-1
	\end{pmatrix} $ 
	sono linearmente indipendenti, dunque dobbiamo controllare se uno dei vettori colonna rimanenti sia linearmente indipendenti con queste. Questo significa calcolare solo due minori invece di quattro.
	
	
	$ 
	\begin{pmatrix}
		3&0&-1\\
		1&2&2\\
		0&-1&-1
	\end{pmatrix}
	=1\ne 0$, quindi $ rank(A)=3 $.
	
\end{es}
%
\end{document}

